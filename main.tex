\documentclass[a4paper]{article}

%% Language and font encodings
\usepackage[english]{babel}
\usepackage[utf8x]{inputenc}
\usepackage[T1]{fontenc}

%% Sets page size and margins
\usepackage[a4paper,top=3cm,bottom=2cm,left=3cm,right=3cm,marginparwidth=1.75cm]{geometry}

%% Useful packages
\usepackage{amsmath}
\usepackage{graphicx}
\usepackage[colorinlistoftodos]{todonotes}
\usepackage[colorlinks=true, allcolors=blue]{hyperref}
\usepackage[newcommands]{ragged2e}

\title{\bf Improving the quality of code search: A tool for accessing suitable reusable eHealth software code\newline \vspace{20pt}\newline}
\author{By:\\\\Mabel Kemigisha\\\vspace{20pt}2017/HD05/891U\\ Master of Science in Data Communication and Software Engineering\\{\bf Option:} Software Engineering \\Makerere University\\School of Computing and Informatics Technology\\P.O.Box 7062 Kampala, Uganda\\\\Mob: +256703141563 | +256774330137\\\\A Research Proposal Submitted to the Directorate of Graduate Research for the study\\ leading to a Dissertation in Partial Fulfillment of the Requirements for the Award of \\the Degree of Master of Science in Data Communication and Software Engineering of\\ Makerere University\\\\{\bf Supervisor:} Benjamin Kanagwa, PhD\\Department of Networks School of Computing and Informatics Technology\\Makerere University\\bkanagwa@cit.mak.ac.ug\\\\ {\bf Co-Supervisor:} Joyce Nabende Nakatumba, PhD\\\\\\\\August, 2018}
\date{}
\newpage


\begin{document}
\maketitle \newpage

{\bf List of Acronyms} \newpage

\tableofcontents
\newpage

\section{INTRODUCTION}
\vspace{10pt}
\subsection{Background to the study}
With the advancement in Information and Communications Technology (ICT), it is evident that all
sectors can and are actually tapping into the benefits that accrue due to the implementation of such
technologies in various organisations around the world. The health sector is one of those that have
advanced in the use ICTs, with many software systems implemented for health facilities around the
world in a bid to ease the management of patients’ medical records as well as their relationship with
medical practitioners and specialists. This is increasingly transforming how health care is delivered,
giving more people in remote areas in Africa and around the world access to better care. Similarly,
easier access to data helps both doctors and policymakers make better-informed decisions about how
to continue to improve the health care system. The use of eHealth systems is not only of benefit to
the healthcare practitioners and patients but also to the vendors. According to Statista [1], a statistics
portal, the revenue projected in the "eHealth" market amounts to United States Dollars (USD) 12,005m in 2018, with most of revenue generated in the United States (USD 3,821m in 2018) and
the rest elsewhere. This clearly shows that vendors (software development companies) of eHealth systems continually reap huge benefits from the market growth yielded by their software products. For any software development project, various stakeholders normally make their contributions in
formulating, analysing, formalising, implementing and testing specifications that the system to be developed (or under development) must possess. Some of the key stakeholder roles include the project sponsor/client, domain experts, project manager, technical lead, software developers, software testers and user acceptance testers among others.\newline
\subsection{The Problem} \newline
For electronic health (eHealth) software systems where records of human life are stored and retrieved, it is very important for software development organisations to put into consideration the factors of integrity and reliability as those of prime importance while building these systems. This is majorly because the patient data stored in such systems is very sensitive nature. Just like paper formats, electronic patients’ medical records provide documented accounts of patients’ medical history, containing information about diagnosis, treatment, consecutive progress notes from doctors as well as discharge recommendations. Using eHealth systems to improve the management of these records is very beneficial to both the health providers and clients, but it can have long-term detrimental effects to patients’ privacy if such information is breached by unauthorised and malicious users, both humans and systems. With any software development project, there is a strong need for timely completion and cost minimisation while ensuring that the system meets all the user requirements, which are functional in nature. As a result, software developers tend to put all their emphasis on ensuring that the specified user requirements are met, although timeliness and budgetary requirements are often indirectly traded in the process. Moreover, factors of utmost importance, that is to say, non-functional requirements such as eHealth software security, privacy and extensibility are often left in the dark side as software developers strive to achieve the required system functionality in the time frame given for system production. Despite software reuse becoming a modern software development practice that software engineers are increasingly employing in order to reduce system development time and minimise software production costs, the practice is not utilised sufficiently because of the vast amount of time that software engineers invest in finding suitable reusable code and utilising it to obtain targeted system functionality. Even after obtaining the code, they do not have sufficient time to study the quality attributes of the code. In summary, there is need for a centralised reference of reusable code that can be applied by software engineers during the development or modification of eHealth applications to enhance productivity and hence, dependability of eHealth systems.\newline 
\subsection{Contribution} \newline 
In this research, we aim at improving the quality of code search during Search-based software engineering and contributing towards the field by developing a search tool that will analyse existing open source eHealth applications/ solutions and identifying suitable reusable code (based on a software developer’s description) that software engineers can plug into their code while developing eHealth systems/applications. With such a centralised reference that will not only enable eHealth system engineers get quick access to suitable reusable code for various aspects of health but also act as an advisory point, we hope that software engineers will be able to complete the production of a fully functional eHealth system with in the speculated time while also making it possible to cut down the development costs. In addition, software engineers will be able to concentrate on the other important system aspects (non-functional system requirements such as system security and privacy) that play a great role in ensuring that patient records are stored and retrieved securely to only intended and authorised users. Also, some of the functional development time saved can be invested in improving the usability of eHealth systems to be deployed and used by health facilities such that they are better utilised in their operational environment.
\subsection{Justification}\\ 
In order to improve health care, many eHealth systems have been developed and are continually being put to use by many health providers around the world, yet the quality of such systems remains compromised due to a number of development constraints. During the development of such systems, software engineers endeavor find code that that they can incorporate into their own software under development to help attain the desired functionality faster than coding from scratch. Even after finding code that can be of help to them through browsing the various open source code search engines in existence, software engineers rarely find time to think and look into the quality issues of the obtained code before trying to incorporate it into the system under development, which in return results in the production of low quality software systems. Because of the sensitive nature of the information stored by eHealth systems, software quality is a paramount issue that should be put into consideration, directly or indirectly, while developing such software. It is therefore necessary to devise a way of providing assistance to software engineers by making the code search activity for customised eHealth software of better quality, ensuring that quality reusable code for eHealth systems is obtained during the code search activity.\\

\subsection{Research Questions}\\ More specifically, the following research questions need to be addressed;\\
RQ1. How can we identify reusable code for eHealth systems?\\
RQ2. How do we effectively store and retrieve such reusable code for eHealth systems?\\
RQ3. How can we conduct a search for open source software in the health domain?\\
RQ4. How do we measure the quality of the obtained code?\\ 

\subsection{Research Objectives}\\
The general objective of the research is to improve the quality of code search during search-based software engineering by coming up with a search tool that will be used as a reference by software engineers to find suitable reusable code (along with advisory information) that they can plug into their code while implementing eHealth software systems to enhance productivity.
Particularly, the study has the following sub-objectives.\\
1. To search for existing Open Source health applications.\\
2. To understand the various aspects of the health domain which are included in eHealth sytems.\\
3. To analyse code in existing eHealth applications and devise a mechanism to extract reusable code for each aspect of health included.\\
4. Identify security loopholes in some of the commonly used eHealth systems.\\
5. To understand the architectures and implementation technology (ies) used in eHealth systems.\\
6. To build a search tool that can filter reusable code in Open Source eHealth applications and return the most suitable code which developers can add to their code.\\
7. To provide recommendations for effective code reuse during implementation of eHealth software.
\\

 \vspace{-10pt}
\section{REVIEW OF RELATED LITERATURE}
\vspace{-10pt}
This chapter will provide detailed review of the literature and/or research related to the study. The
chapter will be divided into sections that include (a) eHealth (b) software reuse (c) storing and retrieving code (d) conducting code search activities (e) Open Source software (f) software quality.


\subsection{Software Reuse}
According to Krueger (1992), software reuse (also known as code reuse) "is the process of creating software systems from existing software rather than building software systems from scratch." It involves the use of some previously constructed software artifacts such as source code, libraries, components and requirements and design documents among others. Long (2017) indicates that software reuse is a productivity technique attempted by many development organizations, with mixed success. Typically, reuse-based software engineering activities include accessing the necessary existing systems to identify
reusable candidates or components, modifying and packaging (where necessary)
them independently, and, understanding the meaning of and other aspects of the reusable artifact before actually reusing it. There are various approaches of software reuse reported in the literature, and they can generally be categorised into component-based software reuse, domain engineering & software product lines, and architecture-based  software reuse [45, 46]. More reuse work has been done in component-based software reuse (CBSR), although all the approaches are mostly used in combination. \\
Component-based Software Engineering (CBSE) is a branch of software engineering that emphasises the separation of concerns in respect to the wide range of the wide –ranging functionality available throughout a given software system. It is a reuse-based approach to defining, implementing and composing loosely coupled components into systems. This practice aims to bring about a wide range of benefits in both the short-term and the long-term for the developed as well as the organisations that sponsor such software. Software engineering practitioners regard components as part of the starting platforms for service orientation. Components play this role, for instance, in web services and more recently in service-oriented architectures, whereby a component is converted by the web service into a service and subsequently inherits further characteristics beyond that of an ordinary component. Components can produce or consume events and can be used for event-driven architectures. An individual software component is a software package, a web service, a web resource, or a module that encapsulates a set of related functions. All system processes are placed into separate components so that all of the data and functions inside each component are semantically related. Because of this principal, it is often said that components are modular and cohesive. With regard to system-wide co-ordination, components communicate with each other through interfaces. When a component offers services to the rest of the system, it adopts a given interface that specifies the services that other components can utilise, and how they can do so. This interface can be regarded as a signature of the component, and the client does not need to know about the inner workings of the component in order to make use of it. This principle results in components referred to as encapsulated. \\
The use of software components enables practitioners to build well-defined interfaces that hide unnecessary implementation details, enhance interoperability with other software components, and ensure the component's ability to be reusable in several different software systems. The major challenge of CBSE/CBSR is managing repository of  a  large  number  of  reusable components  and  developing an  efficient component retrieval mechanism [46]. 
\\ \\
{\bf Methods for identifying reusable code} \\
The ability to identify reusable software components in a timely manner can be of great importance to software engineers and developers. Although effectively accomplishing this is still a challenge, literature shows that there are methods that have been proposed for identifying reusable code from existing systems. These are explained below:\\ \\
\textbf{\textit{Natural Language processing:}}\\
Natural language processing (NLP) is the field of designing methods and algorithms that take as input or produce as output unstructured, natural language data [50]. According to Wikipedia, Natural language processing (NLP) is concerned with the interactions between computers and human (natural) languages, particularly how to program computers to process and analyse large amounts of natural language data. Machine learning, a field of computer science that uses statistical techniques to give computer systems the ability to "learn" with data, without being explicitly programmed, is the primary focus. Some Natural language techniques for software programs or systems are discussed below;\\

\textit{Syntactic Analysis:} The importance of a word token depends both on its type and on the specific linguistic context in which it appears [48]. Syntactic analysis is a computationally efficient first step to identify which words bear contentful information in the document, under the assumption that there is a regular mapping between the content of a text and its syntactic structure. Words can be tagged, heads of phrases extracted and identification of subjects and objects achieved due to the fixed word order of English [48]. Because linguistic heads normally appear in semantically more prominent positions than non-head words, document representation consisting of the linguistic heads of the content words can be used. The heads of the principal content phrases can be individuated in order to reduce the dimensionality of the document representation without losing too much content.\\

\textit{Semantic Analysis:} According to expertsystem.com, semantic analysis is the process of describing the meaning of natural language text, helping machines understand the meaning of the text. From a data processing point of view, semantics are “tokens” that provide context to language. \\

\textit{Parse trees:} According to Wikepedia, a parse tree or parsing tree can be defined as an ordered, rooted tree that represents the syntactic structure of a string according to some context-free grammar. The reflect the syntax of the input language, and they are usually constructed based on either the constituency relation of constituency grammars (phrase structure grammars) or the dependency relation of dependency grammars.  \\

\textit{Neural Networks:} A neural network is a series of algorithms that endeavors to recognize underlying relationships in a set of data through a process that mimics the way the human brain operates. It contains layers of interconnected nodes, and each node is a perceptron and is similar to a multiple linear regression. Neural networks can adapt to changing input so the network generates the best possible result without needing to redesign the output criteria [49].  The two main kinds of neural network architectures that can be combined in various ways include: feed-forward networks and recurrent or recursive networks [50]. \\ \\
Feed-forward networks include networks with fully connected layers, as well as networks with convolutional and pooling layers. All of the networks act as classifiers, each with different strengths. \\
Multi-layer perceptrons (MLPs) are a type of feed-forward networks which allow to work with fixed length inputs, or with variable length inputs in which we can disregard the order of the elements. When feeding the network with a set of input components, it learns to combine them in  a meaningful way. \\
Convolutional feed-forward networks are specialised architectures that excel at extracting local patterns in the data; they are fed arbitrarily sized inputs, and are capable of extracting meaningful local patterns that are sensitive to word order, regardless of where they appear in the input. Hence, they work very well for identifying indicative phrases or idioms of up to a fixed size in long sentences or documents [50]. 
\\ \\
Recurrent neural networks(RNNs) are specialized models for sequential data. They are network components that take as input a sequence of items, and produce a vector that summarizes that sequence. Summarizing a sequence meansdifferent things for differnt tasks. In otherwords, the information needed to answer a question about the sentiment of a sentence is different from the information needed to answer a question about its grammaticality.  Recurrent networks are rarely used as stand alone components, and their power is in being trainable components that can be fed into other network components, and trained to work in tandem with them. Recurrent networks are very impressive models for sequences, and are said to be the most exciting offer of neural networks for language processing [50]. \\ \\
{\bf Storage and retrieval of reusable code} \\






\subsection{Code Search}
Several research works related to code search activities have been conducted by various scholars, contributing to the field of search-based software engineering. They including the following:\\
Mansoor et al. proposed an approach for detection of code-smells using good and bad design examples [].  The authors focused a multi-objective search-based approach for the generation of code-smell detection rules from code-smell and well-designed examples, with an aim of finding the combination of software metrics, from an exhaustive list of metric combinations that maximizes the coverage of a set of code-smell examples collected from different systems, and minimizes the detection of examples of good design practices.\\
Iqbal et al. gave an insight into the design implications for task-specific search utilities for retrieval and re-engineering of code []. The researchers started with a discussion on how software engineers interact with information and the general purpose information retrieval systems, and then investigated to what extent domain-specific search and recommendation utilities can be developed in order to support the work-related activities of software engineers. To investigate this, the researchers conducted a user study through an automated observation to analyse the copy and paste behaviours of software developers.They also discussed the implications for the development of task-specific search and collaborative recommendation utilities embedded with selected general search engines for retrieval and re-engineering of code. Using the feedback obtained during their study, the researchers implemented a prototype of the proposed collaborative recommendation system and evaluated it in a controlled environment.\\
In yet another research, Niu et al. proposed a code example search approach that applies a machine learning technique to automatically train a ranking schema []. The researchers use the trained ranking schema to rank candidate code examples for new queries at run-time, and evaluate the ranking performance of their proposed approach using a variety of code snippets from open-source projects for Android. They identified Some features of selected code examples for queries, and collected the relevances between the queries and the corresponding code examples. They then applied a learning-to-rank algorithm to learn how the identified features should be combined to build a ranking schema.\\
In the area of health, Johnson et al.proposed a code repository for reproducing information on critical health care []. The main focus of the researchers was on providing a description of the MIMIC Code Repository, a centralized location for derived concepts that are relevant to critical care research. Detailed descriptions are provided on how the concepts are defined and extracted from the database, including the assumptions that are made and the conditions for which codes or queries are valid. The study also provides additional tools are provided to educate researchers on best practices for conducting a fully reproducible study using the database.\\\\
Despite the related research works identified above among others having made a great contribution towards the code search activities of software engineers, their main focus is not really on the quality of code search activities in a specific software domain. Because domain expertise, especially in the field of health is very specialised and not generic, coupled with the various flaws introduced during software development activities, it is essential to assist software engineers to build eHealth systems of a higher quality and maintain them in a timely manner with minimal challenges. This research focuses on improving the quality of code search during search-based software engineering, ensuring that software engineers/developers can obtain quality reusable code that they can plug into their code while building eHealth systems.

\subsection{Overview of software applications}
Also known as application software or applications, software applications refer to computer software designed to perform a group of coordinated functions, tasks, or activities for the benefit of the user [10]. Depending on the activity for which it was designed, an application can manipulate text, numbers, graphics, or a combination of these elements. Some application packages focus on a single task whereas others, referred to as integrated software include several applications. Some software applications are available in various versions for a number of different platforms while others only work on one platform.They can be available in form of packaged software, custom software, web application, open source software, shareware, freeware, and public domain software. \\
Software applications can be categorised into the following:\\ Business applications-applications software that assists people while performing business activities. Examples include word processing, databases and business software suite among others.\\
Graphics and multimedia applications- such as computer-aided design software, multimedia authoring and photo editing software.\\
Home/personal/educational applications -such as reference and educational software, entertainments and tax preparation software)\\
Communications applications (such as E-mail, instant messaging, and video conferencing).  

\subsection{eHealth}
The term eHealth has been used and defined by various researchers in several ways in order to provide an understanding within the context of health information. Electronic health, also commonly shortened to e-health or eHealth, is the electronic form of healthcare, and the concept of electronic health is often used alongside electronic health records (EHRs) [19*]. According to Eysenbach [11**], e-health refers to health services as well as information delivered and/or improved through the use of the internet and related technologies. It characterises the technical development, state-of-mind and a commitment for networked, global thinking aimed at improving health care at various levels (locally, regionally, and worldwide) by using information and communications technology (ICT). McLendon (2000) explains that eHealth not only informational but also educational and commercial products to direct services offered by professionals, non-professionals, businesses or consumers themselves, and that it includes includes a wide variety of the clinical activities that have traditionally characterised as tele-health. Some sources [11*, 12*] also point out that eHealth involves increased consumer/patient participation and empowerment even though the consumer may not be the originator of such initiatives. According to the World Health Organization (WHO), eHealth is the cost-effective and secure use of information and communications technologies in support of health and health-related fields, including health-care services,
health surveillance, health literature, and health education, knowledge and research. Generally, WHO describes the primary usage of eHealth in three ways: (i) to disperse healthcare information to patients and clinicians via the web or local network; (ii) to make government health programs better, such as through online training of hospital staff; and (iii) to improve the operation of the healthcare organisation, by healthcare executives. eHealth can include electronic health records (EHRs), or electronic medical records (EMRs) providers as well as software systems or applications [20].\\ \\
{\bf Standards for eHealth}\\
The introduction of digital information and communication technologies did not automatically solve the problems of health [13*]. For instance, those seeking access to medical literature and/or educational health resources had to visit specialised medical libraries even when getting access to medical resources of interest was not guaranteed. Moreover, patient data stored in the files of a primary care physician were not readily accessible by parties (such as specialists and hospitals) that required them because each healthcare provider stored individual data of patients. These methods posed many barriers to receiving optimal healthcare services as patients in remote areas still had no direct access to medical professionals, and medical devices could not be electronically connected from remote locations to advanced medical facilities. Also, each healthcare provider had their own installed systems and technologies, which lacked inter-operability with the systems used by other providers, hence, the need to implement eHealth standards. \\Electronic health standards refer to the specifications that enable interoperability among healthcare-related information and communication technologies and systems made by different providers. Standards are the blueprints that technology developers use to create products that will inherently be compatible with other products adhering to these same standards.  [13*]. Electronic health standardisation is concerned with the principles of information processing, management and governance of information with the provision of solutions for associated problems in the field of health and care [14*]. There are many institutions that deal with standards of standards for eHealth. Some of the primary standards bodies include the following: \\
{\bf ISO/TC 215.} Through its technical commitee 215, the International Organization for Standardization (ISO) establishes eHealth standards in Electronic Health Records.TC 215 has a wide scope which includes “Standardization in the field
of information for health, and Health Information and Communications Technology (ICT) to promote interoperability
between independent systems, to enable compatibility and consistency for health information and data, as well as to reduce duplication of effort and redundancies” [15*]. ISO’s standards in
“Health Informatics” address healthcare delivery, clinical research, public health, and prevention and wellness.The following are the names of some of ISO’s 93 published health information standards: Electronic reporting of adverse drug reactions; Archetype interchange specification; Security; and Interface specification. Part of ISO’s activity in e-health involves the rebranding of specifications developed in other standards-setting institutions such as HL7 or DICOM. For example, ISO 12052:2006 is “Digital Imaging and Communication in Medicine (DICOM)”.\\
{\bf CEN/TC 251.} This is the health informatics technical committee of Comité Européen de Normalisation, the European Committee for Standardization (CEN). It majorly publishes standards that address application and content layer issues in eHealth [16*], such as CEN/TS 15699:2009 “Health informatics – clinical knowledge resources – Metadata” and CEN/TS 15212:2006 “Health informatics – Vocabulary – Maintenance procedure for a web-based terms and concepts database”. Most of CEN's standards are concerned with aspects of information representation, message standards, electronic health records, and some areas of communication specifications between medical devices.\\
{\bf Health Level Seven (HL7).} HL7 is an organisation for standards development, which issues international application layer [17*]
healthcare standards for the electronic exchange and management of health information such as clinical data and administrative information. Although HL7 refers to the standards organization itself, the name is commonly used to refer to specific standards that the institution develops. HL7's work groups cater for various standards, including those that address electronic health records, infrastructure and messaging, as well as imaging integration. The standards are categorised [18] into primary standards, which are considered the most popular standards integral for system integrations, inter-operability and compliance; foundational standards, which define the fundamental tools and building blocks used to build the standards, and the technology infrastructure that implementers of HL7 standards must manage; messaging and document standards for clinical specialties, which are normally implemented once primary standards for the organization are set; Electronic Health Records (EHR) profiles, which provide functional models and profiles that enable the constructs for management of electronic health records; implementation guides, which are created to be used along with existing standards; rules and references, which depict the technical specifications, programming structures and guidelines for software and standards development; education and awareness standards, which grant trial rights as well as helpful resources and tools to enable users understand and adopt HL7 standards. 
Just like most other eHealth standards bodies, HL7 partners with other institutions, such as the International Organization for Standardization (ISO) in issuing international eHealth standards.\\ \\
{\bf What are eHealth applications?} \\
The phrase "eHealth applications" is used to refer to the software and services that manage, transmit, store or record information used in the delivery of healthcare treatment, payment or record keeping [20]. eHealth applications typically use the internet to transmit and store patient data, either for a health service provider or payer. Generally, eHealth applications are used by various health service providers such as doctors and/or medical specialists, hospitals and insurance companies to record patient health information, also known as protected health information (PHI) - any kind of information in a medical record that can be used to identify an individual, including medical records, billing information, health insurance information plus any demographic information for a specific individual [21, 22, 23 ]. The Health Insurance Portability and Accountability Act (HIPAA) sets the standards for protecting sensitive patient data in eHealth applications, and all eHealth aplications that manage PHI are required to be HIPAA compliant [21, 22, 23 ]. to mention but a few, eHealth applications/systems provide various services, such as primary care and specialist referral services, which could involve a primary care or allied health professional providing a consultation with a person, or a specialist assisting a primary care doctor in rendering a diagnosis;remote patient monitoring, which involves using devices to remotely collect and send information to a home health agency or a remote diagnostic testing facility for interpretation; and consumer health and medical information, which involves the use of the Internet and wireless devices for people to obtain specialised health information as well as online discussions and peer-to-peer support [24].\\ \\
{\bf Open Source Vs. Proprietary eHealth applications}\\
Burkett (2016) explains that all kinds of software have some sort of licence, an end-user licence agreement which is created to protect the copyright of the software, and actually restricts the ways in which the end user can utilise that particular software [26]. Although software licensing can be categorised into various licensing models, the terms- prorietary software and open-source software are commonly used to encompass the different categorisations[25, 27]. In that regard, proprietary software involves the owner of the application granting access to their product through an end-user license agreement (also known as EULA), which entails what the customers are entitled to do when using the software, and in which context they are able to use the application [27]. The customer will normally need to agree to the Terms & Conditions of the EULA.\\
Open Source software licenses,  on the other hand work differently and allow software to be used, modified, and shared, without restriction. Users have access to the source code of the application. It is however important to note that although Open-source software is free, not all free software is open-source and can be proprietary [27, 28].\\ \\
Accordingly, both proprietary and open-source eHealth applications are in existence. Unlike  their proprietary eHealth applications counterparts whose usage and adoption practices tend to be slow due to the economic factor (that is to say, they are expensive yet they require long-term maintenance) [29], coupled with the security issues affecting proprietary software [30], open source eHealth software or applications are not only popular for being economical but also for possessing robust features supported by sophisticated and robust technology [31]. \\ \\
{\bf Benefits and drawbacks of using eHealth applications}\\
The usage of eHealth applications in delivering health care is linked to numerous benefits. Some of the main benefits include among others: 
improved access to medical practitioners and/or medical specialists remotely [33, 34]; improved medical diagnosis and hence, reduction in medical errors [34, 38]; improved decision-making [40]; improved efficiency and effectiveness of health care delivery [35, 36]; improved system usability and easy computerisation of health records [34]; and, more record stability due to activity standardisation [37]. Moreso, through decreased paper work, eHealth systems reduce the cost of maintaining patients health records and offer protection for the patient's privacy [39].\\ \\
Although eHealth systems/applications are very beneficial to both the health care provider and the patient, growing research has shown that there are some challenges or drawbacks as well. Those commonly listed include: cost challenges, such as those incurred to purchase required hardware and hiring qualified personnel, the lack of or inadequacy of technical and clinical resources, as well as the difficulty in standardisation of all health information systems [37, 39]; difficulty maintaining the privacy of patient records is also an increasing concern, especially for patients due to the increasing amount of health information exchanged electronically [39, 40]; researchers have also found that many medical conditions do not have scientifically based guidelines for eHealth software vendors to follow while implementing eHealth systems, and hence, reducing their usefulness and effectiveness in many clinical situations [40].\\ 

\subsubsection{Common Open-Source eHealth Applications}
Wikipedia [41] presents a list of over 80 open-source eHealth software, categorised into: Public health and biosurveillance, Electronic records and medical practice management, Health system management, Prescriptions and medicines management, Disease management, Imaging/visualization, Medical information systems, Research, Mobile devices, Out-of-the-box distributions, Interoperability, and Specifications software. Also, the Capterra Medical Blog [44] specifies a "full" list of 326 Electronic Medical/Health Records (EMR-EHR) Software products. The following are indicated as some of the commonly used eHealth applications/products:\\ \\
{\bf OpenMRS} [42, 43] - This a community, platform, application and infrastructure. It can be easily customised to meet specific needs and can handle multiple projects. It is also patient-centric, with limited administrative functionality. Also, OpenMRS requires quite a huge investment of time and energy to create a customised EMR. Moreover, in-depth medical and systems analysis knowledge is important for efficient and effective customisation. \\
{\bf VistA} [42] - VistA HRMS is a complete and comprehensive HR system. It is said to be a reliable and the largest EMR solution, open enough to allow the general super user to make necessary customisations to fit their needs or even for higher level customisations. However, it requires staff to attain adequate training to make the system do what they need. In addition, VistA is not as user-friendly or easy to maintain since it has been in existence for a long time now.\\
{\bf FreeMED} [42, 43] - FreeMED is one of the longest-running open source EMRs. Although FreeMED has over the years had an active support community which contributes to its stability and support/ maintenance, it may require strong technical knowledge to develop or customise the software. \\
{\bf OpenEMR} [42, 43] - OpenEMR is a certified complete web-based ambulatory EMR/EHR. It offers e-prescribing, patient scheduling for multiple facilities, and patient appointment reminders through email and SMS. OpenEMR is known for its reputable support community that is always aif you have any issues or queries. \\
{\bf One Touch EMR} [42] - This is a cloud-based EMR solution that includes electronic prescriptions, lab integration, and a drawing tool for annotations. It is certified for meaningful use and it is easy to use with a quick medical documentation feature. More so, its template library includes multiple specialties. Reviewers however comment that One Touch EMR's learning curve tends to be steep.\\
{\bf NOSH} [42] - NOSH is a Software as a Service (Saas) solution designed for out-patient clinics. It has an intuitive, fast, web-based, modern interface with a patient portal, appointment scheduling system as well as a messaging function. It further offers many templates, electronic forms, electronic order entry, practice management features, graphing, patient-education document creation, alerts, and innovate tagging functions among others. Similar to other open source solution, there’s need for time investment to make necessary configurations and customisations. \\
{\bf Solismed} [42]- Solismed is a medical clinic management system that handles the schedules, bills, utilities, communications as well as supplies of both the patients and the clinic. It stores contact information for patients, Physicians, volunteers and suppliers, and it can track refill requests, open orders, and lab results, and interfaces with numerous public health agencies and immunisation registries. Personnel can also track unpaid invoices and patient payments among other features. Although Solismed’s patient portal is mobile responsive,  its design is not mobile responsive, so it doesn’t display consistently on tablet computers.\\
{\bf HospitalRun} [43] - HospitalRun is built specially to meet the low resource settings of developing countries. It is offline-enabled to allow records to be carried to remote clinics. That is to say, it works when there is no Internet, and syncs when there is. HospitalRun offers great user experiences for clinicians, administrators, and even contributors, and its features are optimized to save time.\\
{\bf Bahmni} [43] - This is yet another hospital system designed for low resource settings. It is said to be reliabe and can be configured easily to suit a hospital's work flow. In addition, it combines and enhances existing open source products into a single solution.\\
{\bf Cottage Med} [43] - This is an EMR designed by physicians for physicians. It is accessible and Affordable, easily customisable and expandable.\\ \\

{\bf GNUmed} [43] - \\
{\bf OpenClinic} [43] - \\
{\bf OpenEyes} [43] - \\
{\bf WorldVistA} [43] - \\
{\bf openMAXIMS} [43] - \\
{\bf Open Hospital} [43] - \\
{\bf GNU Health} [43] - \\
{\bf FreeMedForms} [43] - \\
{\bf ZEPRS} [43] - \\
{\bf SMART Pediatric Growth Chart} [43] - \\
{\bf LibreHealth EHR [43]} - \\
{\bf THIRRA} [43] - \\
{\bf FreeHealth.io EHR} [43] - \\




\subsection{Software Quality}

\subsubsection{Overview of Software quality}

\subsubsection{Factors that affect software quality}

\subsubsection{Software quality metrics}



\section{RESEARCH METHODOLOGY}
\vspace{-10pt}
The major paradigm for this research will be Design Science and Action Research. Herein below, we present our proposed methodology that we plan to use to address our research questions.\\ \\
\textit{RQ1. How can we identify reusable code for eHealth systems?} \\
In our approach, we envision the identification phase of reusable code for eHealth systems in two perspectives- researcher perspective and system perspective. The first perspective is that of the researcher, which will focus on ensuring that the researcher understands electronic health information and is able to identify such aspects while analysing eHealth source code. The second perspective, which is the system perspective, will focus on ensuring that our proposed tool is able to understand natural language module descriptions supplied by the software engineers.  \\ \\
\textit{\textbf{Researcher perspective:}} From reviewing preliminary literature, we have obtained some knowledge about electronic health. Going forward, we propose to perform further domain analysis by reviewing documented technical literature about electronic health. In so doing, we will mostly rely on SNOMED CT (Systematised Nomenclature of Medicine Clinical Terms)'s library of medicine as a base for eHealth information. According to SNOMED  International [], SNOMED CT's library is the most comprehensive, multilingual clinical healthcare terminology in the world; it has scientifically validated clinical content and that is mapped to other international standards. Other online health libraries including MEDLINE shall also be consulted. \\
Having a clear understanding of health information provided in the said sources will aid the researchers in understanding the various aspects of the health domain that can be implemented in electronic health systems. Moreover, our ability to understand the code blocks  in identified eHealth software source code that support specific health care aspects, will be enhanced. \\
Then, with help of the in-depth knowledge obtained from technical health literature, and using a crawler, various source code in online repositories shall be sampled and observation made both physically and with software tools that support pattern identification during code analysis.
 \\ \\
\textit{\textbf{System perspective:}} From the viewpoint of the proposed tool, identification of eHealth code from the pool of source code provided will be the first step towards retrieval of the appropriate reusable code. We propose to use the Natural Language Processing (NLP) approach identified in the literature. A  search function shall be designed and implemented with a linkage to the SNOMED CT library such that module descriptions supplied by the software engineer or developer can be matched with identifiers and comments basing on the international health terminologies provided by SNOMED and such other libraries. We choose to use the identifier and comments approach because in object-oriented code, which is typically organised in classes [], an understanding of such code can be readily achieved by analysing the comments and identifiers that are added to the code blocks. While analysing the comments and identifiers, we will follow a top-down approach, which involves analysing the base classes, followed by derived classes defined in the code base. While implementing SNOMED CT in our proposed tool, we plan to employ the natural language processing techniques for using SNOMED CT proposed by confluence.ihtsdotools.org [47]. They include: Stemming- the process of reducing a word to its stem, base or root form (for instance, "cardiology", "cardiac" and "cardiologist" may be reduced to the stem "cardi"); Reordering- the process of reordering the words in a phrase (for instance, reordering "hip fracture" to "fracture hip"); Word substitution- the process of substituting a word or word phrase with an equivalent word or word phrase; and,Stop word removal- the process of removing words with limited semantic specificity (such as 'a', 'an', 'and', 'as', 'at', 'be', 'by', 'for', 'of', 'the', etc.). \\ \\
\\\textit{RQ2: How can we effectively store and retrieve reusable code for eHeath?} \\
We do not intend to use machine-dependent storage for this study; online source code repositories will be major storage facilities and available source code synchronised with the search tool to be designed. To derive answers for this research question, code snippets shall be prepared. In order to prepare code snippets for open source eHealth applications, several source code repositories shall be crawled or searched to find code with strings relevant to the health domain. To extract code snippets from open source eHealth software, a relevant parser will be used to extract a code snippet from each method defined in the source code files with a specified file extension (s).\\ \\
Due to the nature of open source source software, searching and getting access to open source code (RQ3) can be achieved by first identifying the available repositories/online collections of open source code. Literature about existing open source repositories will be reviewed and the identified population sampled based on their relevance to the health domain.\\ \\
Measuring the quality of the obtained code (RQ4) will be done through experimentation. The design process of the proposed search tool will be performed with a set of activities that will be designed to be circular or spiral in nature; these activities will be developed and implemented in such a way that the quality aspect of the code search gradually improves through the various iterations of the modifications/ changes to the tool. During the experiments, relevant defect detection methods and techniques shall be employed on the eHealth source code to be fetched by the code search tool proposed. Prior to returning source code, eHealth code detected by the tool shall be checked for its adherence to architectural and coding standards. A rating (also based on best known code standards and practices) will then be attached to each returned code snippet. Basing on the assigned ratings and using a code ranking algorithm, will be organised and displayed in ascending order of their quality competence.  \\ \\This study will be conducted between September and December 2018.
\vspace{-10pt}
\section{REFERENCES}
\vspace{-10pt}
[1] Statista, eHealth. Retrieved on July 18, 2018, from:
\href{https://www.statista.com/outlook/312/100/ehealth/worldwide#market-users}{Statista}\\
[2][2]\\
[3][3]\\
[4][4]\\
[5][5]\\
[6][6]\\
[7][7]\\
[8][8]\\
[9][9]\\
[10][10]Wikipedia, Application Software. Retrieved on August 27, 2018 from: \href{https://en.wikipedia.org/wiki/Application_software}{Wikipedia}\\
[11][11]Eysenbach G. What is e-health? J Med Internet Res. 2001 Jun;3(2):e20.\\
[12][12] Doherty I. A Movement Within The Health Community. Health Care and Informatics Review Online 2008;12(2):49-57.\\
[13][13]ITU-T Technology Watch Report (2012). E-health Standards and Interoperability.\\
[14][14]EU, CEN/TC 251 Health Informatics. Retrieved on September 11, 2018 from: \href{http://www.ehealth-standards.eu}{EU}\\
[15][15]\\ISO/TC 215, Health Informatics. Retrieved on September 11, 2018 from: \href{https://www.iso.org/committee/54960.html}{ISO}\\
[16][16]ISO/TC 215, “Health Informatics”, scope available at URL ()
www.iso.org/iso/iso_technical_committee?commid=54960. \\
[17][17]HL7 Standards,About HL7. Retrieved on September 11, 2018 from: \href{http://www.hl7.org/about/index.cfm?ref=nav}{HL7}\\
[18][18]HL7 Standards,Introduction to HL7 Standards. Retrieved on September 11, 2018 from: \href{http://www.hl7.org/implement/standards/index.cfm?ref=nav}{HL7 Standards}\\
[19][19]e-Health. Retrieved on September 11, 2018 from: \href{https://www.atlantic.net/hipaa-compliant-hosting/what-are-e-health-applications/}{Atlantic}\\
[20][20]eHealth applications. Retrieved on September 11, 2018 from: \href{https://www.truevault.com/ehealth.html}{Truevalt}\\
[21][21]Wikipedia, Protected health information. Retrieved on September 10, 2018 from: \href{https://en.wikipedia.org/wiki/Protected_health_information}{Wikipedia}\\
[22][22]Indiana University, What is protected health information (PHI)?. Retrieved on September 10, 2018 from: \href{https://kb.iu.edu/d/ayyz}{Indiana University}\\
[23][23]TechTarget, Protected health information (PHI) or personal health information. Retrieved on September 10, 2018 from: \href{https://searchhealthit.techtarget.com/definition/personal-health-information}{TechTarget}\\
[24][24]Disabled-world, Services provided by telemedicine. Retrieved on September 11, 2018 from: \href{https://www.disabled-world.com/medical/ehealth/}{Disabled-world}\\
[25][25]Monitis, Software Licensing Types Explained. Retrieved on September 11, 2018 from: \href{http://www.monitis.com/blog/software-licensing-types-explained/}{Monitis}\\
[26][26]WorkingMouse. WHAT IS SOFTWARE LICENSING? Retrieved on September 11, 2018 from: \href{https://workingmouse.com.au/innovation/software-licensing-why-its-important-and-how-it-can-help-you}{WorkingMouse}\\
[27][27]Quora. Software Licensing. Retrieved on September 11, 2018 from: \href{https://www.quora.com/What-are-different-types-of-software-licenses}{Quora}\\
[28][28]pen Source Initiative. Introduction. Retrieved on September 11, 2018 from: \href{https://opensource.org/osd-annotated}{Open Source Initiative}\\
[29][29]S. Bowen, R. Hoyt, G. Ladeana, D. McCormick, and X. Gonzalez. Open-source electronic health records: Practice implications, 2011. \\
[30][30]J. Carl Reynolds and C. Jeremy Wyatt. Open source, open
standards and health care information systems, 2011, J Med
Internet Res, 13(1):e24.\\
[31][31]N. AISSAOUI, M. AISSAOUI, and Y. JABRI. For a Cloud computing based Open source E-Health
Solution for Emerging Countries, 2013, International Journal of Computer Applications, 84(1).\\
[32][32]M. Botha,  A. Botha, and M. Herselman. The Benefits and Challenges of e-Health Applications: A Content Analysis of the
South African context, 2015.\\
[33][33]S. F. Fontenot. The Affordable Care Act and
Electronic Health Care Records: Does today’s
technology support the vision of a paperless health
care system?, PEJ, pp. 72–76, 2013.\\
[34][34]D. Lobach and D. Detmer. Research challenges for
electronic Health Records, American Journal of
Preventive Medicine, 32 (5S), pp. 104–111, 2007.\\
[35][35]N. Calman, D. Hauser, J. Lurio, W. Y. Wu, and M.
Pichardo. Strengthening Public Health and Primary Care Collaboration Through Electronic Health Records, American Journal of Public Health, vol. 102, no. 11, pp. e13–e18, 2012. \\
[36][36]L. Kern M., A. Edwards, and R. Kaushal, “The
Patient-Centered Medical Home, Electronic Health
Records, and Quality of Care,” Annals of Internal
Medicine, vol. 160, no. 11, pp. 741–754, 2014.\\
[37][37]S. Jarosławski and G. Saberwal, “In eHealth in India
today, the nature of work, the challenges and the
finances: an interview based study,” BMC Medical
Informatics and Decision Making, vol. 14, no. 1,
2014.\\
[38][38]M. B. Buntin, M. F. Burke, M. C. Hoaglin and D. Blumenthal. The Benefits Of Health Information Technology, Health Affairs, 30, no.3 (2011):464-471 \\
[39][39] Advantages & Disadvantages Of Electronic Health Record Retrieved on September 11, 2018 from: \href{https://www.e-spincorp.com/2017/11/23/advantages-disadvantages-of-electronic-health-record/}\\
[40][40]N. Menachemi and T. H. Collum. Benefits and drawbacks of electronic health record systems. Retrieved on September 12, 2018 from: \href{https://www.ncbi.nlm.nih.gov/pmc/articles/PMC3270933/}\\
[41][41]Wikipedia. List of open-source health software.  Retrieved on September 12, 2018 from: \href{https://en.wikipedia.org/wiki/List_of_open-source_health_software/}\\
[42][42]Capterro Medical SOftware Blog, The Top 7 Free and Open Source EMR Software Products. Retrieved on September 12, 2018 from: \href{https://blog.capterra.com/top-7-free-open-source-emr-software-products/}\\
[43][43]Top 20 FREE and Open Source EMR - EHR. Retrieved on September 12, 2018 from: \href{http://medevel.com/top-20-free-and-open-source-emr-ehr/}\\
[44][44]Capterro Medical SOftware Blog, Electronic Medical Records (EMR) Software. Retrieved on September 12, 2018 from: \href{https://www.capterra.com/electronic-medical-records-software/}\\
[45][45]T. Hemmann. Reuse Approaches in Software Engineering and
Knowledge Engineering: A Comparison, 2014. \\
[46][46]Researchgate, A REVIEW OF APPROACHES TO SOFTWARE REUSE. Retrieved on September 14, 2018 from: \href{https://www.researchgate.net/publication/263926582/download/}\\
[47][47]NLP Techniques using SNOMED CT. Retrieved on September 18, 2018 from: \href{https://confluence.ihtsdotools.org/display/DOCANLYT/5.1+Natural+Language+Processing/}\\
[48][48]T. P. Merlo, J. Henderson, G. Schneider and E. Wehrli (Geneva). Learning Document Similarity
Using Natural Language Processing. \\
[49][49]Investopedia, Neural Nework. Retrieved on September 15, 2018 from: \href{https://www.investopedia.com/terms/n/neuralnetwork.asp}\\
[50][50]Y. Goldberg. Neural Network Methods for Natural Language Processing, 2017. \\



\end{document}